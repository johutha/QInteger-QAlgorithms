\begin{titlepage}
    \centering
    {\huge \textbf{Faktorisieren auf Quantencomputer \newline Erklärung und Implementation}}
    \vspace{1cm}
    \newline
    {\LARGE\textit{Erklärung der Funktionsweise von Quantencomputern und Implementation zweier Programmbibliotheken in Q\# mit arithmetischen Opertationen sowie einer vollständigen Implementation von Shors Algorithmus}}
    \vspace{1cm}
    \newline
    {\Large \begin{tabular}{l@{}ll}
        Maturaarbeit von \; \; &  Joël Benjamin Huber & \\
        Betreut von & Christian Steiger & \\
        an der & Kantonsschule Freudenberg Zürich & \\
    \end{tabular}
    }
    \vspace{3cm}
    \newline
    {

        \textbf{Zusammenfassung: } \textit{Diese Maturitätsarbeit beschäftgt sich mit Quantencomputern und deren Anwendungen für mathematische Probleme, spezifisch das Faktorisieren von Zahlen. Dafür wurden arithmetische Operationen genauer angeschaut, um zu verstehen, wie jene auf Quantencomputern implementiert werden können. Danach wird Shors Algorithmus genauer betrachtet, mit welchem sich Zahlen faktorisieren lassen. Dies alles wurde als zwei Programmbibliotheken in der Quantenprogrammiersprache Q\# und der klassischen Programmiersprache C\# implementiert, sodass man diese Operationen in anderen Projekte benutzen kann. }
    }
\end{titlepage}