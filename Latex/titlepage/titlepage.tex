\begin{titlepage}
    \centering
    {\huge \textbf{Faktorisierung auf dem Quantencomputer\newline Erklärung und Implementierung}}
    \vspace{1cm}
    \newline
    \begin{minipage}[c]{01\textwidth}
    {\LARGE\textit{Einführung in die Funktionsweise von Quantencomputern und Implementation zweier Programmbibliotheken in Q\# mit arithmetischen Operationen sowie einer vollständigen Implementierung des Shor-Algorithmus}}
    \end{minipage}
    \vspace{1cm}
    \newline
    {\Large \begin{tabular}{l@{}ll}
        Maturaarbeit von \; \; &  Joël Benjamin Huber & \\
        Betreut von & Christian Steiger & \\
        an der & Kantonsschule Freudenberg Zürich & \\
        abgegeben am & 15. Dezember 2020 & \\
    \end{tabular}
    }
    \vspace{3cm}
    \newline
    \begin{minipage}[c]{1\textwidth}
    {
        \textbf{Zusammenfassung: } \textit{Die vorliegende Maturitätsarbeit beschäftgt sich mit Quantencomputern und ihrer Anwendung für das Faktorisierungsproblem. Zu diesem Zweck wurden arithmetische Operationen genauer betrachtet, um zu verstehen, wie sie auf Quantencomputern implementiert werden können. Danach wurde der Shor-Algorithmus genauer betrachtet, mit welchem sich Zahlen faktorisieren lassen. Dies alles wurde in der Form zweier Programmbibliotheken in der Quantenprogrammiersprache Q\# und der klassischen Programmiersprache C\# implementiert, sodass man diese Operationen auch in anderen Projekten verwenden kann. }
    }
    \end{minipage}
\end{titlepage}