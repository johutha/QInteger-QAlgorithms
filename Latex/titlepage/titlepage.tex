\begin{titlepage}
    \centering
    {\begin{fullwidth}[leftmargin=-2cm, rightmargin=-2cm, width=\linewidth+4cm]
        \centering
        \huge \hspace{1cm} \textbf{Faktorisierung auf dem Quantencomputer\newline Einführung und Implementierung}
    \end{fullwidth}}
    \vspace{1cm}
    \begin{minipage}[c]{01\textwidth}
    {\LARGE\textit{Einführung in die Funktionsweise von Quanten-computern und Implementierung zweier Programm-bibliotheken in Q\# mit arithmetischen Operationen und einer vollständigen Implementierung des Shor-Algorithmus}}
    \end{minipage}
    \vspace{1cm}
    \newline
    {\Large \begin{tabular}{l@{}ll}
        Wettbewerbsarbeit von \; \; &  Joël Benjamin Huber & \\
        Betreut von & Christian Steiger & \\
        an der & Kantonsschule Freudenberg Zürich & \\
        eingereicht am & 28. Oktober 2021 & \\
    \end{tabular}
    }
    \vspace{3cm}
    \newline
    \begin{minipage}[c]{1\textwidth}
    {
        \textbf{Zusammenfassung: } \textit{Die vorliegende Wettbewerbsarbeit beschäftigt sich mit Quantencomputern und ihrer Anwendung für das Faktorisierungsproblem. Zu diesem Zweck habe ich arithmetische Operationen genauer betrachtet, um zu verstehen, wie sie auf Quantencomputern implementiert werden können. In einem weiteren Schritt habe ich mich mit dem Shor-Algorithmus auseinandergesetzt, mit welchem sich Zahlen faktorisieren lassen. Dies alles habe ich in der Form von zwei Programmbibliotheken in der Quantenprogrammiersprache Q\# und der klassischen Programmiersprache C\# implementiert, mit dem Ziel, dass man diese Operationen später auch in anderen Projekten verwenden kann. }
    }
    \end{minipage}
\end{titlepage}