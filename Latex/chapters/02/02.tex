\chapter{Grundlagen}
\section{Lineare Algebra}

\section{Quantensysteme}
Die Quantensysteme, die wir im Bereich des Quantum Coputing anschauen, sind rein mathemathematische Systeme, die auf keiner fixen physikalische Realisierung basieren. Dies bedeutet, dass es verschiedene Implementierungen gibt, die sich je nach Situation besser oder weniger gut eignen.

\textbf{Definition 2.2.1:} Ein \textbf{Qubit} ist das kleinste Quantensystem und damit die kleinste Informationseinheit in einem Quantencomputer. Das System hat die beiden Basiszustände $\ket{0}$ und $\ket{1}$ und kann somit alle Zustande $\alpha\ket{0} + \beta\ket{1}$ mit $\abs*{\alpha}^2 + \abs*{\beta}^2 = 1$ annehmen.

Den Zustand $\alpha\ket{0} + \beta\ket{1}$ kann man auch mit einem Vektor $\begin{bmatrix}\alpha \\ \beta\end{bmatrix}$ darstellen. Die Bedingung, dass $\abs*{\alpha}^2 + \abs*{\beta}^2 = 1$ gelten muss, bedeutet, dass der Vektor normiert sein muss. Die Chance, den Zustand $\ket{0}$ zu messen, $\abs*{\alpha}^2$, und für den Zustand $\ket{1}$ gleich $\abs*{\beta}^2$ ist. Hier sehen wir wieder, dass der Vektor normiert sein muss, denn sonst würden sich die Wahrscheinlichkeiten nicht auf 1 summieren.
Genereller lassen sich Zustände, die $n$ Qubits involvieren, als Vektoren von Grösse $2^n$ darstellen. Sei $a_j$ der $j$-te Eintrag in jenem Vektor. Die Chance, den Zustand $j$ zu messen, wobei das $k$-te Qubit dann im Zustand $\lfloor \frac{j}{2^k} \rfloor \pmod{2}$ ist (was einfach dem $k$-ten Bit in der Binärdarstellung von $j$ entspricht), ist dann $\abs*{\alpha_j}^2$. Auch hier summieren sich die Wahrscheinlichkeiten auf 1, da der Vektor normiert ist. Man kann diese Zustände aber auch in Form eines mathematischen Ausdrucks $\alpha_0\ket{0} + ... + \alpha_j\ket{j} + ... + \alpha_{2^n - 1}\ket{2^n - 1}$ oder $\alpha_{00\dots}\ket{000\dots} + ... + \alpha_{j}\ket{j} + ... + \alpha_{111\dots}\ket{111\dots}$ darstellen, wobei in der zweiten Schreibweise $j$ ein Bitstring von Länge $n$ ist.

Gehen wir zurück zum einzelnen Qubit im Zustand $\alpha\ket{0} + \beta\ket{1}$. Wir wissen schon, dass wir bei einer Messung mit einer Wahrscheinlichkeiten von $\abs*{\alpha}^2$ den Zustand $\ket{0}$ messen. Nach dieser Messung kollabiert das Quantensystem in den gemessenen Zustand. Das heisst, messen wir den Zustand $\ket{0}$, befindet sich das Quantensystem nachher im Zustand $\ket{0}$, egal, wie $\alpha$ und $\beta$ vorher waren. Dies gilt auch für Multi-Qubit Systeme. Diese kollabieren dann in die noch möglichen Quantenpositionen. Nehmen wir als Beispiel ein 2-Qubit-System im Zustand $\frac{1}{\sqrt{6}}\ket{00} + \frac{1}{\sqrt{2}}\ket{01} + \frac{1}{\sqrt{3}}\ket{11}$ und messen das erste Qubit. Die Chance, dass wir dieses Qubit im Zustand $\ket{1}$ messen, liegt bei $\abs*{\alpha_{10}}^2 + \abs*{\alpha_{11}}^2 = \frac{1}{3}$. Falls wir diesen Zustand messen, kollabiert unser Quantensystem sofort in den Zustand $\frac{\alpha_{10}\ket{10} + \alpha_{11}\ket{11}}{\sqrt{\abs*{\alpha_{10}}^2 + \abs*{\alpha_{11}}^2}} = \ket{11}$, wobei die Summe im Nenner dafür sorgt, dass der neue Quantenzustand wieder normalisiert ist. Die Chance eines $\ket{0}$ in der Messung des ersten Qubit hingegen liegt bei $\abs*{\alpha_{00}}^2 + \abs*{\alpha_{01}}^2 = \frac{2}{3}$. Der Zustand des Systems nach der Messung ist dann $\frac{\alpha_{00}\ket{00} + \alpha_{01}\ket{01}}{\sqrt{\abs*{\alpha_{00}}^2 + \abs*{\alpha_{01}}^2}} = \sqrt{\frac{1}{4}}\ket{00} + \sqrt{\frac{3}{4}}\ket{01}$.
Dass diese Zustände rein mathematisch durch Vektoren darstellen lassen, hat zur Folge, dass man Quanten verschränken kann. Dafür schauen wir uns den einfach realisierbaren Zustand $\frac{1}{\sqrt{2}}(\ket{00} + \ket{11})$. Nun platzieren wir das eine Qubit auf die eine Seite des Universums, und das andere Qubit auf die andere Seite. Dann messen wir das erste Qubit. Die beiden Zustände $\ket{0}$ und $\ket{1}$ haben dabei die gleiche Wahrscheinlichkeit. Das andere Qubit auf der anderen Seite des Universums, kollabiert darauf sofort in den gleichen Zustand wie das erste. Ich habe dabei die beiden Seiten des Universums gewählt, um darauf aufmerksam zu machen, dass diese Qubits miteinander verknüpft sind, und dabei physikalische Distanz keine Rolle spielt.