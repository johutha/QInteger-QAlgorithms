\chapter{Arithmetische Operation auf Qubits ausführen - Die QInteger Library}
\section{Überblick}

\section{Zahlen in Qubits speichern - Der QInt-Typ}
Da ich davon ausgehe, dass in nächster Zeit die Anzahl Qubits zwar wachsen, aber nicht so schnell ansteigen wird, dass man schon bald mehrere grössere Qubit-Einheiten speichern kann, habe ich mich entschieden, in meiner Implementation auf eine einheiltiche Grösse zu verzichten und dafür auf eine variable Grösse zu setzen. Deshalb besteht der QInt-Typ aus einer klassischen Zahl, der Anzahl Qubits, und einem Array von Qubits, welcher die eigentliche Zahl speichert. Ich habe mich auch dazu entschieden, die Quantenzahl im Little-Endian Format zu speichern, da so neue Qubits einfach angehängt werden können, ohne den Wert der Zahl zu verändern. 

\lstinputlisting[language=C++]{assets/code/3/3.2.1}

\section{Die Quanten-Fouriertransformation und die Fourier-Basis}
Die Quanten-Fouriertransformation ist eine Transformation, die eine Quantenzahl von der uns bekannten binären Basis in die Fourierbasis transformiert. Die Fouriertransformation, die dabei auf den Qubits implementiert ist, ist mathematisch definiert als eine Transformation, die aus einem Vektor $(x_0, x_1, ..., x_{n - 1})$ den Vektor $(y_0, y_1, ..., y_{n - 1})$ macht, mit $y_k = \frac{1}{\sqrt{n}}\sum_{j = 0}^{n - 1}x_je^{2i\pi\frac{kj}{n}}$. Da dies ein linearer Operator ist, genügt es, wenn wir uns die Wirkung des Operators auf die Basiszustände anschauen. Schauen wir also die Wirkung des Operators auf den Basiszustand $\ket{x}$ an. Wir erhalten:
$$QFT\ket{x} = \frac{1}{\sqrt{2^n}}\sum^{2^n - 1}_{j = 0}e^{2i\pi\frac{xj}{2^n}}\ket{j}$$
Gleichzeitig lässt sich dieser Zustand faktorisieren, nämlich zu:
\begin{align*}
\left(\ket{0} + e^{2i\pi\frac{x}{2^{n}}}\ket{1}\right)\otimes\left(\ket{0} + e^{2i\pi\frac{x}{2^{n - 1}}}\ket{1}\right)\otimes\cdots\otimes\left(\ket{0} + e^{2i\pi\frac{x}{2^{1}}}\ket{1}\right) = \\ \bigotimes_{j = 0}^{n - 1}\left(\ket{0} + e^{2i\pi\frac{x}{2^{n - j}}}\ket{1}\right)
\end{align*}
Dies kann man durch ausmultiplizieren beweisen. Um die folgende Gleichung zu vereinfachen, sei hier $b_k(j) = 1$, falls das $k$-te Bit von $j$ gesetzt ist, und $b_k(j) = 0$, falls nicht. Dazu sei $B_j$ als das Set aller $k \in \mathbb{N}_0$ mit $b_k(j) = 1$. Dann bekommen wir:
\begin{align*}
\frac{1}{\sqrt{2^n}}\bigotimes_{j = 0}^{n - 1}\left(\ket{0} + e^{2i\pi\frac{x}{2^{n - j}}}\ket{1}\right) = \frac{1}{\sqrt{2^n}}\sum_{j = 0}^{2^n - 1}\left(\prod_{k \in B_j}e^{2i\pi\frac{x}{2^{n - k}}}\right)\ket{j} \\ \\
= \frac{1}{\sqrt{2^n}}\sum_{j = 0}^{2^n - 1}e^{2i\pi\sum_{k = 0}^{n - 1}\left(\frac{x \cdot b_k(j)}{2^{n - k}}\right)}\ket{j} = \frac{1}{\sqrt{2^n}}\sum_{j = 0}^{2^n - 1}e^{2i\pi\frac{x\sum_{k = 0}^{n - 1}\left(2^k\cdot b_k(j)\right)}{2^n}}\ket{j} \\ \\
= \frac{1}{\sqrt{2^n}}\sum_{j = 0}^{2^n - 1}e^{2i\pi\frac{xj}{2^n}}\ket{j}
\end{align*}
Was bringt uns diese Faktorisierung? Zuerst stellen wir fest, dass die Bits unabhängig und nicht verschränkt sind. Zudem sehen wir, dass der Zustand $\frac{1}{\sqrt{2}}(\ket{0} + e^{2i\pi\theta}\ket{1})$ in der Blochkugel einem Zeiger in der $XY$-Ebene, welcher um $\theta$ um die $Z$-Achse gedreht ist, entspricht. Schauen wir uns die einzelnen Qubits an, entspricht das $j$-te Qubit einem Zeiger in der $XY$-Ebene, gedreht um $\frac{x}{2^{n - j}}$ um die $Z$-Achse. Dies ist die sogenannte Fourier-Basis.

% TODO Insert Picture of "Blochkugel"

Die Fourierbasis hat verschiedene Vorteile. Zum Beispiel werden wir im Kapitel 4.3 sehen, dass wenn wir in einer Operation ein Qubit um $\theta$ um die Z-Achse drehen, was der Multiplikation des Koeffizienten von $\ket{1}$ mit dem Wert $e^{2i\pi\theta}$ entpricht, wir nichts anderes tun, als den Wert der Qubits in der Fourierbasis zu verändern. Später können wir dann die inverse $QFT$ anwenden, um den Wert $\theta$ als Binärzahl auslesen zu können. Auch dass die Qubits unabhängig sind, ist ein grosser Vorteil, wie wir bei der Addition feststellen werden.

Schauen wir uns an, wie man die Transformation implementieren kann. Sie lässt sich mit $\mathcal O(n^2)$ Gatteroperationen ohne zusätzliche Qubits implementieren. Schauen wir uns nochmals die Faktorisierung an. Wir stellen fest, dass das letzte Qubit in der binären Basis nur das erste Qubit in der Fourierbasis beeinflusst, da $2^{n - 1}$ in $x$ im Term $e^{2i\pi\frac{x}{2^a}}$ mit $a \leq n - 1$ nur ganze Rotationen im Einheitskreis hinzufügt, was den Wert nicht beeinflusst. Weiterhin beeinflusst das zweitletzte Qubit in der binären Basis nur die beiden ersten Qubits in der Fourierbasis etc. Kehren wir die Reihenfolge der Qubits der Fourierbasis um. Wir bekommen $\bigotimes_{j = 0}^{n - 1}\left(\ket{0} + e^{2i\pi\frac{x}{2^{j + 1}}}\ket{1}\right)$. Dann beeinflusst jedes Qubit in der binären Basis nur das gleiche und alle nachfolgenden Qubits in der Fourierbasis. Nun können wir die Qubits von hinten nach vorne in die Fourierbasis transformieren und danach die Reihenfolge zurückkehren, was sich ganz einfach mit dem $SWAP$-Operator bewerkstelligen lässt. Betrachten wir nun das Qubit $j$. Alle Qubits nach $j$ sind schon in der Fourierbasis und alle vorher noch nicht. Zuerst wenden wir den $H$-Operator auf das Qubit an. Wir erhalten den Zustand $\frac{1}{\sqrt{2}}(\ket{0} + \ket{1})$, falls $b_j(x) = 0$ und $\frac{1}{\sqrt{2}}(\ket{0} - \ket{1})$, falls $b_j(x) = 1$. Dies ist nichts anderes als $\frac{1}{\sqrt{2}}(\ket{0} + e^{2i\pi\frac{b_j{x}2^j}{2^{j + 1}}}\ket{1})$, was den Beitrag vom $j$-tzen Qubit in der binären Basis an das $j$-te Qubit in der umgekehrten Fourierbasis ist. Nun müssen wir nur noch den Beitrag aller Qubits vor dem $j$-ten Qubit dazurechnen. Das $l$-te Qubit mit $l < j$ soll $e^{2i\pi\frac{b_l(x)2^{l}}{2^{j + 1}}}$ zum Koeffizienten beitragen. Das $b_l(x)$ lässt sich so umsetzen, dass wir das $l$-te Qubit in der binären Basis als Kontroll-Qubit für die Operation nehmen, die $e^{2i\pi\frac{2^{l}}{2^{j + 1}}}$ zum Koeffizienten von $\ket{1}$ dazurechnet. Dies ist nichts anderes als das $Rot(k)$-Gatter, mit $k = (j + 1) - l$. Damit haben wir eine Implementation für die Quanten-Fouriertransformation, die $\mathcal O(n^2)$ Gatteroperationen und keine zusätzlichen Qubits benötigt.

\lstinputlisting[language=C++]{assets/code/3/3.3.1}

\section{Addition}
Die wohl grundlegendste arithmetische Operation ist die Addition. Die Subtraktion kann als Addition ausgedrückt werden und auch die Multiplikation (und somit die Division) ist abhängig von der Addition. Deshalb ist sie die erste arithmetische Operation, die wir betrachten. 

Wir wollen die Operation auf zwei QInts implementieren, welche zwei QInts im Zustand $(\ket{x}, \ket{y})$ in den Zustand $(\ket{x}, \ket{x + y})$ umwandelt. Die Implementation anderer Additionsmethoden (Addition einer klassischen Zahl zu einem QInt, Addition zweier QInts in ein drittes QInt) funktionieren analog. Zusätzlich kann man auch sehen, dass die Subtraktion nichts anderes als die inverse Operation zur Addition ist, somit hat man zur Addition die Subtraktion mit-implementiert.

Für die Addition gibt zwei verschiedene Techniken, die oft benutzt werden. Die eine benutzt zusätzliche Carry-Bits und erreicht so eine Gatterzahl in $\mathcal O(n)$, benötigt aber $\mathcal O(n)$ zusätzliche Qubits, während die andere ohne zusätzliche Qubits auskommt, dafür aber $\mathcal O(n^2)$ Gatteroperationen benötigt. Ich habe mich entschieden, die zweite Version in meiner QInteger-Library zu implementieren. Gründe dafür sind, dass in heutigen Systemen die Anzahl verfügbarer Qubits stark begrenzt ist und in Simulationen einzelne Qubits sehr viel zusätzliche Leistung benötigen, während eine Laufzeit von $\mathcal O(n^2)$ in diesem Fall weniger ausmacht. Sobald mehr Qubits zur Verfügung stehen, wird es lohnenswerter, auf die andere Version zu wechseln, denn da Addition eine Operation auf einem sehr tiefen Level ist, kann die Zeit, welche die Addition benötigt, sehr grosse Auswirkungen auf die gesamte Laufzeit haben.

Schauen wir uns nun den in der QInteger-Library verwendeten Additionsalgorithmus an. Der Algorithmus basiert auf der Fourierbasis (und damit auf der Faktorisierung der Fouriertransformation). Bei der Addition in der binären Basis sind die einzelnen Bits voneinander abhängig. Deshalb werden sogenannte Carry-Bits verwendet, welche für jedes Bit abspeichern, ob wir beim nächsten Bit noch ein zusätzliches $1$ addieren müssen. Dies ist bei der Fourierbasis nicht so: Die Bits sind voneinander unabhängig. Das heisst, wir können die einzelnen Bits voneinander unabhängig modifizieren, ohne dabei auf die anderen Bits achten zu müssen. Dies ist der grosse Vorteil der Fourier-Basis, welcher uns erlaubt, auf zusätzliche Qubits zu verzichten. Betrachten wir uns nochmals die Faktorisierung an: Das $j$-te Qubit der Zahl $y$ in der Fourierbasis ist im Zustand $\frac{1}{\sqrt{2}}(\ket{0} + e^{2i\pi\frac{y}{2^{n - j}}}\ket{1})$. Wir wollen es aber in den Zustand $\frac{1}{\sqrt{2}}(\ket{0} + e^{2i\pi\frac{x + y}{2^{n - j}}}\ket{1})$ bringen, denn wenn wir alle Qubits in den entsprechenden Zustand bringen könnten, könnten wir mit der inversen $QFT$ den Zustand $\ket{x + y}$ wiederherstellen. Nehmen wir wieder das aus der Fouriertransformation bereits bekannte Gatter $Rot(k) = \frac{1}{\sqrt{2}}\begin{bmatrix}
    1 & 0 \\
    0 & e^{\frac{2i\pi}{2^k}} \\
\end{bmatrix}$. Mit dem Gatter können wir den Wert $2i\pi\frac{1}{2^k}$ dem Exponenten von $\ket{1}$ hinzufügen. Das heisst, wenn wir das Gatter auf ein Qubit im Zustand $\frac{1}{\sqrt{2}}(\ket{0} + e^{2i\pi\frac{y}{2^{n - j}}}\ket{1})$ anwenden, wird es in den Zustand $\frac{1}{\sqrt{2}}(\ket{0} + e^{2i\pi\frac{y + 2^{n - j - k}}{2^{n - j}}}\ket{1})$ versetzt. Wir können also mit Hilfe dieses Gatters Zweierpotenzen zum Qubit in der Fouriertransformation addieren. Wenn wir das Qubit im Zustand $\ket{x}$ in der binären Basis lassen, können wir die Addition wie folgt mit $\mathcal O(n^2)$ Gatteroperationen implementieren:
\begin{enumerate}
    \item Wende $QFT$ auf den zweiten Summanden im Zustand $\ket{y}$ an. Das Register befindet sich nun im Zustand $\frac{1}{\sqrt{2^n}}\bigotimes_{j = 0}^{n - 1}\left(\ket{0} + e^{2i\pi\frac{y}{2^{n - j}}}\right)$.
    \item Für das jedes $j$-te Bit im zweiten Register, wende für jedes $k$-te Bit im ersten (binären) Register mit $k < n - j$ ein kontrolliertes $Rot(n - i - j)$ an. Das $j$-te Bit befindet sich nachher im Zustand 
    \begin{align*}
        \frac{1}{\sqrt{2}}(\ket{0} + e^{2i\pi\frac{y + \sum_{k = 0}^{n - j - 1}{b_k(x)\cdot 2^{n - j - (n - j - k)}}}{2^{n - j}}}\ket{1})
        = \frac{1}{\sqrt{2}}(\ket{0} + e^{2i\pi\frac{y + x}{2^{n - j}}}\ket{1})
    \end{align*}
    Wobei alle Bits höher als $2^{n - j - 1}$ uns nicht interessieren, da sie alle Vielfache von $2^{n - j}$ sind und somit nur ganze Umrundungen zur Rotation hinzufügen.
    \item Die Qubits im zweiten Register befinden sich nun in folgendem Zustand: $\frac{1}{\sqrt{2^n}}\bigotimes_{j = 0}^{n - 1}\left(\ket{0} + e^{2i\pi\frac{x + y}{2^{n - j}}}\right)$. Mit der inversen $QFT$ kann man nun aus diesem Zustand den Zustand $\ket{x + y}$ wiederherstellen.
\end{enumerate}
\section{Modulare Addition}
Den uns bekannten Modulus-Operator kann man auf Qubits nicht implementieren, da er nicht reversibel ist ($a$ und $a + m$ haben dasselbe Resultat Modulo $m$). Die modulare Addition ist jedoch reversibel, wenn beide Summanden kleiner als das Modulo sind, denn wir können die Informationen über die beiden Summanden benützen. Wir haben wieder zwei Register in den Zuständen $\ket{x}, \ket{y}$ und eine klassische Zahl $m$, und möchten die Register in den Zustand $\ket{x}, \ket{(x + y) \% m}$ setzen. Hier lässt sich das Modulo $m$ durch einen QInt ersetzen (oder den ersten Summanden durch eine klassische Zahl). Für Shors Algorithmus benötigen wir die Operation nur mit einem klassischen Modulo $m$, die Implementation für QInt-Modulos folgt analog und sind auch in der QInt-Library enthalten.

Zuerst addieren wir $\ket{x}$ zum Register $\ket{y}$, um das Register in den Zustand $\ket{x + y}$ zu versetzen. Nun überprüfen wir, ob diese Summe grössergleich dem Modulo $m$ ist. 

Wie überprüfen wir, ob eine Zahl grössergleich einer anderen Zahl ist? Sagen wir, ob $\ket{A}$ grössergleich der Zahl $B$ ist, wobei $B$ auch einen QInt sein könnte? In der QInteger-Library ist die Funktion $GreaterOrEqual$ für $A >= B$ als ¬$LessThan$ implementiert. Hier unterscheiden sich die Implementationen für de Fälle wenn $B$ ein QInt oder eine klassische Zahl ist, sie machen jedoch das Gleiche. Wir schauen uns die Implementation für den Fall an, wenn $B$ eine klassische Zahl ist. Wir wissen, dass $A >= B$ gilt (für $A$ und $B$ ganze Zahlen), falls $A - B < 0$ oder $A - B == 0$ gilt. Da nicht beide Bedingungen gleichzeitig erfüllt sein können, können wir die Resultate der beiden Checks einfach auf das gleiche Qubit setzen. Ein QInt ist genau dann gleich 0, wenn alle seine Qubits auf 0 sind. Gleichzeitig, falls das Resultat der Subtraktion kleiner als $0$ sein soll, gibt es einen Underflow, das heisst, die Zahl $A - B$ wird zu $2^n - (B - A)$, und somit das grösste Qubit auf 1 gesetzt. Das heisst, wir können einfach das grösste Qubit betrachten. Es kann aber passieren, dass $A - B >= 2^{n - 1}$ gelten kann, deshalb verlängere ich in meiner Implementation das Register, welches $A$ enthält, um 1.

\lstinputlisting[language=C++]{assets/code/3/3.5.1}

Nun können wir messen, ob $x + y >= m$ gilt, und diese Information in einem zusätzlichen Qubit speichern. Falls $x + y >= m$ gilt, subtrahieren wir $m$ von der Zahl und haben somit den Zustand $\ket{x + y - m}$ im zweiten Register. Nun ist die Information, ob $x + y >= m$ gilt, noch in einem Qubit gespeichert, welches wir zurücksetzen müssen. Hier machen wir die Beobachtungen, dass $x + y >= m$ genau dann gilt, wenn das Resultat grössergleich dem Summanden $x$ ist. Die Richtung $res < x \rightarrow x + y >= m$ ist nicht schwierig. Für die andere Richtung sehen wir, dass $x + y - m >= x$ genau dann gelten kann, falls $y >= m$ gilt, was aber nach der Annahme $x, y < m$ nicht stimmen kann. Somit können wir mit diesem Vergleich die Information in unserem Aushilfsqubit wieder löschen.

\section{Modulare Multiplikation}
Mit Hilfe der Modularen Addition können wir nun die modulare Multiplikation implementieren. Zuerst stellen wir fest, dass wir die Moudulare Multiplikation $\ket{x} \rightarrow \ket{(ax) \pmod{m}}$ nur dann implementieren können, wenn $ggT(a, m) = 1$ gilt, denn sonst wäre sie nicht reversiel. \newline
Wir schauen uns die modulare Multiplikation auf QInts in zwei Schritten an.

Zuerst implementieren die Quantenoperation auf zwei Register, welche für gegebenes $a$ und $m$ folgende Operation implementiert: $$U'_{a, m}\ket{x}\ket{y} = \ket{x}\ket{(y + ax) \pmod{m}}$$.
Dafür sehen wir, dass wenn wir $x = 2^0 x_0 + 2^1 x_1 ...$ in seine Zweierpotenzen aufteilen, dann können wir $ax = x_0(2^0 a \pmod{m}) + x_1(2^1 a \pmod{m}) + ...$ schreiben. Dieses $+ \;x_0(...)$ ist nichts anders als Addition, kontrolliert durch das $x_0$ Qubit. Dies heisst, wir können diese Operation relativ einfach durchführen: \newline
Für jedes $j$, führe eine modulare Addition, kontrolliert durch das Qubit $x_j$, auf das Ausgaberegister mit dem Summanden $2^ja \pmod{m}$ durch, den wir klassisch berechnen können. \newline
Diese Unteroperation ruft den modularen Addierer $\mathcal O(n)$ mal auf und jede dieser Additionen braucht $\mathcal O(n^2)$ Gatteroperationen. Damit kommen wir auf $\mathcal O(n^3)$ Gatteroperationen.

Mit Hilfe dieser Unteroperation können wir nun die Operation, welche $$U_{a, m}\ket{x} = \ket{(ax) \pmod{m}}$$ bewirkt, implementieren:
\begin{enumerate}
    \item Führe ein temporäres Register im Zustand $\ket{0}$ ein, und bringe es mit Hilfe der oberen Unteroperation den Zustand $\ket{ax \pmod{m}}$.
    \item Berechne klassisch das Inverse von $a$ Modulo $m$. Dieses Inverse existiert, da $a$ und $m$ teilerfremd sind.
    \item Wende die inverse Operation der oben definierten Unteroperation mit $a^{-1}$ mit dem temporären Register als Eingaberegister und dem ersten Register als Ausgaberegister an. Dies ergibt dann den Zustand $$\ket{x - a^{-1}(ax) \pmod{m}}\ket{ax \pmod{m}} = \ket{0}\ket{ax \pmod{m}}$$.
    \item Wechsle den Wert der beiden Register mit Hilfe der Swap-Oeration, wir bekommen den Zustand $\ket{ax \pmod{m}}\ket{0}$. Das temporöre Register im Zustand $\ket{0}$ können wir wieder freigeben und das erste Register ist nun im Zustand $\ket{ax \pmod{m}}$.
\end{enumerate}
Diese Multiplikation benötigt $n$ extra Qubits für das temporäre Register. Sie ruft die Unteroperation $2 \in \mathcal O(1)$ mal auf, und benötigt somit $\mathcal O(n^3)$ Gatteroperationen. Wir erinnern uns daran, dass man die Addition mit $O(n)$ Gatteroperationen und dafür $n$ zusätzlichen Qubits implementieren kann, was dazu führt, dass wir nur noch $O(n^2)$ Gatteroperationen benötigen, dafür aber $2n$ extra Qubits benötigen.