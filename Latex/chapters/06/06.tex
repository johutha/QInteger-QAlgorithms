\chapter{Nachwort}
Sind Quantencomputer die Computer der Zukunft? Ich habe dieses nicht ganz einfache Thema gewählt, weil es mich seit längerem fasziniert. Da sie den komplexen, für uns nur schwer zu verstehenden Gesetzen der Quantenmechanik ausgesetzt sind, erlauben Quantencomputer es uns, manches effizienter zu berechnen, was auf klassischen Computern nicht effizient berechenbar ist.  Wenn wir diese Gesetze verstehen und uns zunutze machen können, können wir sehr leistungsfähige Maschinen bauen.
\paragraph{} 
\noindent Das Gebiet der Quantencomputer ist ein sehr junges Forschungsgebiet. Obwohl es sich aktuell rasant weiterentwickelt, ist dieses neue Gebiet uns noch weitgehend unbekannt und es liegt noch sehr viel Potenzial für zukünftige Entwicklungen darin verborgen.
\paragraph{}
\noindent Ich habe in dieser Arbeit versucht, mich der Thematik anzunähern und zu verstehen, was hinter diesen als so leistungsfähig gepriesenen Maschinen steckt. Der Weg, den ich in diesem Projekt zurückgelegt habe, war anspruchsvoll. Es dauerte seine Zeit, bis ich verstand, nach welchen Gesetzen die Quanten sich verhalten und welche mathematischen Strukturen dahinterstecken, da diese so fundamental anders funktionieren als in der klassischen Welt.
\paragraph{}
\noindent Auch das Programmieren meiner Bibliotheken gestaltete sich ungewohnt. Wenn man auf einem klassischen Computer beispielsweise eine Funktion einer Zahl programmiert,  kann diese Zahl auf einem Quantencomputer verschiedene Werte gleichzeitig haben. Dadurch war es schwierig den Überblick über die Quantenzustände zu wahren und der Prozess des Debuggings wurde sehr zeitintensiv. Auch die Tatsache, dass alle auf Quantencomputern implementierbaren Funktionen invertierbar sein müssen, war für mich ungewohnt. 
\paragraph{}
\noindent Doch die Mühe hat sich für mich gelohnt. Ich habe auf diesem Weg sehr viel gelernt und ich bin zuversichtlich, dass mich dieses Thema auch in Zukunft begleiten wird.
