\chapter{Der Weg zum Shor-Algorithmus}
\section{Überblick}
Im folgenden Kapitel werden wir uns die relevanten Konzepte und Ideen hinter dem quantenbasierten Teil des Shor-Algorithmus anschauen. Dabei beginnen wir beim simplen Konzept des \grqq Phase Kickback\grqq{} , schauen uns dann die darauf basierende Phasenabschätzung an, bevor wir mit deren Hilfe die Ordnung einer Zahl finden. Zum Schluss werden wir uns den kompletten quantenbasierten Teil des Algorithmus zusammenfassend anschauen. Dieses Kapitel basiert hauptsächlich auf \cite{QC} und \cite{QK}.

\section{Phase Kickback}
Ich beginne den Abschnitt mit einer Frage: Wenn wir eine kontrollierte Operation ausführen, sollte sich das Kontrollqubit eigentlich nicht ändern, oder? Im Folgenden werden wir sehen, dass dem überraschenderweise doch so sein kann. Dazu schauen wir uns das CNOT-Gatter an. Was geschieht, wenn wir CNOT auf zwei Qubits im Zustand $\ket{+-}$ anwenden\footnote{$\ket{+}$ und $\ket{-}$ sind im Kapitel 2.2.4 definiert.}, mit dem ersten Qubit als Kontrollqubit? Unser Ausgangszustand ist $\ket{+-} = \frac{1}{2}(\ket{00} - \ket{01} + \ket{10} - \ket{11})$. Nachdem wir das CNOT angewendet haben, erhalten wir den Zustand $\frac{1}{2}(\ket{00} - \ket{01} - \ket{10} + \ket{11}) = \ket{--}$. Überraschenderweise stellen wir fest, dass sich das Kontrollqubit verändert hat, während das Ziel-Qubit gleich geblieben ist. Was ist geschehen? Betrachten wir das CNOT-Gatter genauer: Es ist nichts anderes als eine kontrollierte Version des $X$-Gatters. Was geschieht, wenn wir das $X$-Gatter auf den $\ket{-}$-Zustand anwenden? $X\ket{-} = -\frac{1}{\sqrt{2}}\ket{0} + \frac{1}{\sqrt{2}}\ket{1} = -\ket{-} = (-1)*\ket{-}$. Hier können wir sehen, dass $\ket{-}$ ein Eigenvektor des $X$-Gatters mit Eigenwert $-1$ ist. Der Zustand ändert sich somit bis auf einen globalen Faktor nicht, was eine Unterscheidung zum ursprünglichen Zustand durch eine Messung verunmöglicht, wie in 2.2.2 ausgeführt wurde.

\paragraph{}

Wenn wir hingegen die Operation kontrolliert durchführen, wird diese Phase nur in den Zuständen sichtbar, in welchen die Operation durchgeführt wurde, sprich in den Zuständen, in denen das Kontrollqubit im Zustand $\ket{1}$ ist. Dies konnten wir zuvor beim CNOT-Gatter beobachten. Betrachten wir nun eine allgemeinere Operation $U$ mit einem Eigenvektor $\ket{\psi}$ und dem Eigenwert $\lambda$. Nehmen wir jetzt ein Kontrollqubit im Zustand $\alpha\ket{0} + \beta\ket{1}$, ein Qubit-Register im Zustand $\ket{\psi}$ und wenden $U$ auf das Register $\ket{\psi}$ an, kontrolliert durch jenes Kontrollqubit:
\begin{align*}
(\alpha\ket{0} + \beta\ket{1})\ket{\psi} \xrightarrow{U^c} \alpha\ket{0\psi} + \beta\ket{1}\otimes U\ket{\psi} = (\alpha\ket{0} + \lambda\beta\ket{1})\ket{\psi}
\end{align*}
Das Ziel-Qubit verändert sich nicht – es ist ja ein Eigenvektor – stattdessen sehen wir, dass der Eigenwert zurück in die Phase des Kontrollqubit \grqq gekickt\grqq{} wird, daher der Name \grqq Phase Kickback\grqq{}. Im nächsten Abschnitt werden wir diesen Effekt anwenden, um den Eigenwert eines Operators abzuschätzen.

\section{Phasenabschätzung}
Verschiedene Quanten-Algorithmen basieren darauf, den Eigenwert eines Operators $U$ zu einem Eigenvektor $\ket{\psi}$ abzuschätzen. Dazu werden Phase Kickbacks benutzt, um den Eigenwert in ein Zähler-register in der Fourier-Basis zu schreiben, welches wir dann mit der inversen Quantenfourier-transformation in die binäre Basis zurückrechnen. Dabei können wir die Anzahl Qubits variieren, um die Präzision der Approximation festzulegen. Genauer gibt der Algorithmus zum Eigenwert $\lambda = e^{2i\pi\theta}$ die Zahl $2^n\theta$ zurück, wobei $n$ die Anzahl Qubits des Zählerregisters ist, die für eine bessere Präzision erhöht werden kann. 
\paragraph{}
\noindent Um zu verstehen, wie dieser Algorithmus funktioniert, erinnern wir uns zuerst daran, wie eine Zahl in der Fourierbasis aussieht. Dazu benutzen wir die Bloch-Kugel. Wir erinnern uns, dass für die Zahl $x$ in der Fourierbasis mit $n$ Qubits das $k$-te Qubit (hier muss man aufpassen, denn wenn man die Faktorisierung anschaut, ist das $k$-te Qubit das $k$-te Qubit von rechts für $k = 0, \dots, n - 1$) um $2\pi\frac{2^kx}{2^n}$ um die Z-Achse gedreht wird. Das heisst, es befindet sich im Zustand $\frac{1}{\sqrt{2}}(\ket{0} + e^{2i\pi\frac{2^kx}{2^n}}\ket{1})$. Wir machen jetzt die Beobachtung, dass wir mit Hilfe des Phase Kickbacks das gesuchte $\theta$ in der Fourierbasis in die Kontrollqubits schreiben können, da der Phase Kickback nichts anderes macht, als das Kontrollqubit auf diese Art und Weise zu rotieren. Beobachten wir nun, was geschieht, wenn wir $2^k$-mal kontrolliert den Operator $U$ anwenden (wobei hier $(U^c)^{2^k}$ die kontrollierte Version von $U$ bedeutet, $2^k$-mal angewendet):
\begin{align*}
\frac{1}{\sqrt{2}}(\ket{0} + \ket{1})\ket{\psi} \xrightarrow{(U^c)^{2^k}} &\frac{1}{\sqrt{2}}(\ket{0\psi} + \ket{1}\otimes U^{2^k}\ket{\psi}) \\ = &\frac{1}{\sqrt{2}}(\ket{0} + (e^{2i\pi\theta})^{2^k}\ket{1})\ket{\psi} \\
= &\frac{1}{\sqrt{2}}(\ket{0} + e^{2i\pi2^k\theta}\ket{1})\ket{\psi} \\
= &\frac{1}{\sqrt{2}}(\ket{0} + e^{2i\pi\frac{2^k(2^n\theta)}{2^n}}\ket{1})\ket{\psi}
\end{align*}
Dies entspricht genau dem $k$-ten Qubit der Repräsentation von $2^n\theta$ in der Fourierbasis. Das heisst, wenn wir ein Zählerregister im Zustand $\ket{0}$ nehmen und dann für $k = 0, ..., n - 1$ zuerst das $H$-Gatter auf das $k$-te Qubit und dann $2^k$-mal ein kontrolliertes $U$ mit dem $k$-ten Qubit als Kontrollqubit auf $\ket{\psi}$ anwenden, erhalten wir das $k$-te Qubit der Darstellung von $2^n\theta$ in der Fourierbasis. Wenden wir anschliessend die inverse Fouriertransformation an, können wir die Zahl $2^n\theta$ im Zählerregister ablesen. Falls $2^n\theta$ keine ganze Zahl ist, erhalten wir im Zählerregister eine Superposition, wobei eine Zahl wahrscheinlicher ist, je näher sie beim echten Eigenwert liegt.

\paragraph{}

Um die Phase abzuschätzen, müssen wir also den Operator $U$ mehrmals hintereinander anwenden, zuerst nur einmal, dann zweimal, im $i$-ten Mal $2^i$ mal. Dies führt dazu, dass wir die Operation $\mathcal O(2^n)$ mal anwenden müssen. Allerdings ist es oft möglich, dass wir die Operation $U^{2^m}$ für einen beliebigen Parameter $m$ implementieren können. Wenn dies möglich ist, brauchen wir nur $n$ Anwendungen jener Operation.

\paragraph{Algorithmus}
\begin{enumerate}
    \item Initialisiere zwei Quantenregister, das Zählerregister und das Eigenvektor-Register, und setze das Eigenvektor-Register auf den gewünschten Eigenvektor $\ket{\psi}$.
    \item Wende $H^{\otimes n}$ auf das Zähler-Register an, um es auf $\ket{+}^{\otimes n}$ zu setzen.
    \item Für $k = 0, \dots, n - 1$: Wende auf den Eigenvektor $\ket{\psi}$ die Operation $(U^{c_k})^{2^k}$, also die zur $2^k$-fachen Anwendung von $U$ äquivalente Operation, kontrolliert durch das Qubit $c_k$, an.
    \item Wende die inverse Quantenfouriertransformation auf das Zählerregister an, um die Approximation in die binäre Basis umzurechnen.
    \item Miss das Zählerregister, um die Abschätzung abzulesen.
\end{enumerate}
\paragraph{}
\input{./assets/circuits/4.3.1.crct}
\captionof{figure}{Schaltkreis der Phasenabschätzung}
\paragraph{}

\section{Periodenabschätzung}
Gegeben sei eine Funktion $f : S \rightarrow S$ für eine endlichen Menge $S \subset \mathbb{Z}$, welche sich auf einem Quantencomputer implementieren lässt, und ein Wert $x \in S$. Wir versuchen nun, die kleinste Zahl $r \in \mathbb{N}$ zu berechnen, sodass $f^r(x) = x$ gilt.
\paragraph{}
Da die auf Quantencomputern berechneten Operationen unitär sind, müssen die Funktionswerte von $f$ anhand von invertierbaren Operationen berechnet werden. Deshalb muss $f$ injektiv sein und somit müssen $|S|$ verschiedene Funktionswerte von $f$ existieren, woraus die Surjektivität folgt. Somit ist $f$ bijektiv und man kann $f$ als Permutation der Elemente von $S$ interpretieren. Diese Permutation kann man als Komposition disjunkter Zykeln darstellen\footnote{Für eine genauere Beschreibung und einen Beweis verweise ich auf Seiten 23 ff. von \cite{perm}}.

\vspace{0.1cm}
\hspace{0.8cm}\begin{minipage}[t]{13.3cm}

Ein \emph{Zykel} ist eine Permutation $\sigma : S\rightarrow S$ mit den folgenden Eigenschaften:

\vspace{0.1cm}
Es gibt eine Teilmenge $\{x_1, x_2, \ldots, x_r\}=:S_\sigma \subset S $, so dass

\vspace{0.1cm}
$\begin{array}{ll}

1. & \sigma (x_r)=x_1 \\

2. & \sigma (x_i)=x_{i+1} \text{ \;für $i<r$}  \\

3. & \sigma (x)=x \text{ \; für $x\in S\setminus S_\sigma$}

\end{array}$

\vspace{0.1cm}
Zwei Zykeln $\sigma$ und $\delta$ heissen \emph{disjunkt}, wenn die Mengen $S_\sigma$ und $S_\delta$ disjunkt sind.

\end{minipage}


\vspace{0.2cm}
Seien nun $\sigma_0, ..., \sigma_{k - 1}$ disjunkte Zykeln mit $f=\sigma_0\circ\sigma_1\circ \ldots\circ \sigma_{k - 1}$. Dazu sei $S_i$ die Menge der Zahlen $j \in S$, so dass $\sigma_i(j) \neq j$ gilt.
\paragraph{}
Sei nun $x \in S_i$. Da $\sigma_i$ einen Zykel bildet, gilt $f^{|S_i|}(x) = x$. Gleichzeitig kann kein $r \in \mathbb{N}$ mit $r < |S_i|$ existieren, sodass $f^r(x) = x$ gilt, denn sonst hätte unser Zykel nur $r < |S_i|$ Elemente. Wir wollen nun also für ein $x \in S_i$ die Grösse $|S_i|$ finden.

\paragraph{}
\begin{wrapfigure}{l}{0.5\textwidth}
\input{./assets/graphs/4.4.1.grph}
\caption{\small Der Funktionsgraph der Funktion $g(x) = -x^3 + 1 \pmod{11}$}
\end{wrapfigure}

Als Beispiel nehmen wir $g : S \rightarrow S$ mit $S = \mathbb{Z}/11\mathbb{Z}$, $g(x) = -x^3 + 1$. Wenn wir die Funktionswerte von $g$ betrachten, sehen wir, dass diese Funktion bijektiv ist. Wenn wir dann den Graphen betrachten, sehen wir die einzelnen Zykeln und ihre dazugehörigen Mengen $S_i$: $S_0 = \{0, 1\}$, $S_1 = \{2, 4, 3, 7, 10\}$, $S_2 = \{5, 8, 6\}$ und $S_3 = \{9\}$. Wir sehen nun, dass $g^1(9) = 9$, $g^3(8) = 8$, $g^5(2) = 2$ etc. 

\paragraph{}

Es fragt sich nun, wie wir effizient die Grösse der Teilmenge finden können, in der $x$ sich befindet. Sei dafür $U_f$ der Operator, welcher $f$ auf Quantencomputern implementiert. Wir haben seine Existenz vorausgesetzt. Was geschieht, wenn wir diesem Operator eine Superposition der Zahlen in $S_i$ übergeben? Seien $r = |S_i|$, $x_0, x_1, ..., x_{r - 1}$ die Zahlen in $S_i$, sodass $f(x_j) = x_{(j + 1) \pmod{r}}$ und $U_f$ die Quantenoperation, die $f$ implementiert. Betrachten wir, was geschieht, wenn wir $U_f$ auf den Zustand $\frac{1}{\sqrt{r}}\sum_{j = 0}^{r - 1}\ket{x_j}$ anwenden. Wir erhalten:
\begin{align*}
    U_f(\frac{1}{\sqrt{r}}\sum_{j = 0}^{r - 1}\ket{x_j}) = \frac{1}{\sqrt{r}}\sum_{j = 0}^{r - 1}\ket{f(x_j)} = \frac{1}{\sqrt{r}}\sum_{j = 0}^{r - 1}\ket{x_{(j + 1) \pmod{r}}} = \frac{1}{\sqrt{r}}\sum_{j = 0}^{r - 1}\ket{x_j})
\end{align*}
Daraus schliessen wir, dass $\frac{1}{\sqrt{r}}\sum_{j = 0}^{r - 1}\ket{x_j}$ ein Eigenvektor von $U_f$ mit Eigenwert $1$ ist. Dieser Eigenwert ist nicht wirklich interessant. Wir können ihn aber interessanter machen, indem wir den einzelnen Summanden eine Phase mitgeben. Dazu konstruieren wir die Superposition $\frac{1}{\sqrt{r}}\sum_{j = 0}^{r - 1}(e^{-2i\pi\frac{kj}{r}}\ket{x_j})$ für ein $k < r$. Was geschieht, wenn wie $U_f$ darauf anwenden? 
\begin{align*}
    U_f(&\frac{1}{\sqrt{r}}\sum_{j = 0}^{r - 1}(e^{-2i\pi\frac{kj}{r}}\ket{x_j})) = \frac{1}{\sqrt{r}}\sum_{j = 0}^{r - 1}(e^{-2i\pi\frac{kj}{r}}\ket{x_{(j + 1) \pmod{r}}}) \\ =
    &\frac{1}{\sqrt{r}}\sum_{j = 0}^{r - 1}(e^{-2i\pi\frac{k(j - 1)}{r}}\ket{x_j}) = e^{2i\pi\frac{k}{r}}(\frac{1}{\sqrt{r}}\sum_{j = 0}^{r - 1}(e^{-2i\pi\frac{kj}{r}}\ket{x_j}))
\end{align*}
Hier haben wir es mit einem Eigenvektor zum Eigenwert $e^{2i\pi\frac{k}{r}}$ zu tun. Dieser ist für uns interessant, weil $r$ darin vorkommt. Wir machen die Beobachtung, dass unser Eigenvektor von vorher ($\frac{1}{\sqrt{r}}\sum_{j = 0}^{r - 1}\ket{x_j}$) ebenfalls von der Form ist, die wir gerade analysiert haben, einfach mit $k = 0$. Wenn es uns nun gelingt, einen Zustand in der Form $\frac{1}{\sqrt{r}}\sum_{j = 0}^{r - 1}(e^{-2i\pi\frac{kj}{r}}\ket{x_j})$ zu erzeugen, können wir mit Hilfe der Phasenabschätzung den Quotienten $\frac{k}{r}$ abschätzen. Es fragt sich jedoch, wie wir einen solchen Zustand generieren können. Sei nun $\ket{\psi_k} := \frac{1}{\sqrt{r}}\sum_{j = 0}^{r - 1}(e^{-2i\pi\frac{kj}{r}}\ket{x_j})$. Was geschieht, wenn wir die Vektoren $\ket{\psi_0}, \ket{\psi_1}, ..., \ket{\psi_{r - 1}}$ aufsummieren?
\begin{align*}
    \frac{1}{\sqrt{r}}\sum_{k = 0}^{r - 1}\frac{1}{\sqrt{r}}(\sum_{j = 0}^{r - 1}(e^{-2i\pi\frac{kj}{r}}\ket{x_j})) = \frac{1}{r}\sum_{j = 0}^{r - 1}\sum_{k = 0}^{r - 1}(e^{-2i\pi\frac{kj}{r}}\ket{x_j}) = \ket{x_0}
\end{align*}
Um dieses überraschende Resultat zu beweisen, betrachten wird die Summe $\sum_{k = 0}^{r - 1}(e^{-2i\pi\frac{kj}{r}}\ket{x_j})$ für $j = 0$. Da $j = 0$ gilt, gilt $e^{-2i\pi\frac{kj}{r}} = e^{0} = 1$ und somit $\sum_{k = 0}^{r - 1}(e^{-2i\pi\frac{k\cdot 0}{r}}\ket{x_0}) = r\ket{x_0}$. Da $\frac{1}{r}(r\ket{x_0})$ bereits einen Betrag von $1$ hat, kann kein anderer Zustand mit positivem Betrag existieren, da die Beträge sich sonst zu etwas Grösserem als $1$ aufsummieren würden.

\paragraph{}

Nehmen wir nun die kürzere Schreibweise $\ket{\psi_k} = \frac{1}{\sqrt{r}}\sum_{j = 0}^{r - 1}(e^{-2i\pi\frac{kj}{r}}\ket{x_j})$ für die oben genannten Eigenvektoren wieder auf. Somit ist $\ket{x_0}$ einfach eine Superposition jener Eigenvektoren. Wir zeigen nun, wie man aus dieser Superposition mit Hilfe der Phasenabschätzung den Eigenwert von einem der Zustände $\ket{\psi_k}$ abschätzen kann und wie man aus diesem Eigenwert eine Abschätzung der Periode $r$ erhält.

\paragraph{}
Betrachten wir dazu den Phasenabschätzungs-Algorithmus nochmals bis zum Zeitpunkt vor der Messung des Zählerregisters. Dies können wir als eine Operation $P$ mit $P\ket{0}\ket{\psi} \rightarrow \ket{2^m\theta}\ket{\psi}$ zu einem Eigenvektor $\ket{\psi}$ schreiben, wobei $m$ die gewählte Genauigkeit der Phasenabschätzung ist. Da die auf Quantencomputern implementierten Operationen linear sind, gilt: $$P\ket{0}\ket{x_0} = P\left(\ket{0}\left(\sum_{k = 0}^{r - 1}\ket{\psi_k}\right)\right) = \sum_{k = 0}^{r - 1}\left(P\ket{0}\ket{\psi_k}\right) = \sum_{k = 0}^{r - 1}\left(\ket{2^m\theta_k}\ket{\psi_k}\right) = \sum_{k = 0}^{r - 1}\left(\ket{2^m\frac{k}{r}}\ket{\psi_k}\right)$$

Messen wir dann das Resultat, so kollabiert diese Superposition und wir erhalten den Wert $2^m\frac{k}{r}$ für ein $k \in \{0, \dots r - 1\}$, wobei alle Werte die gleiche Wahrscheinlichkeit haben, mit einer Genauigkeit von $m$ Bits. Mit Hilfe von Kettenbrüchen\footnote{Auf eine geneauere Betrachtung von Kettenbrüchen verzichte ich hier, sollte der Leser damit nicht vertraut sein, verweise ich auf \cite{Frac}} können wir nun den Bruch $\frac{k}{r}$ vom Bruch $\frac{2^m\frac{k}{r}}{2^m}$ abschätzen. Nun stellt sich heraus: Wenn $m \geq 2\log(|S|) + 1$ gilt, so ist der Bruch $\frac{2^m\frac{k}{r}}{2^m}$ exakt genug, sodass der Bruch mit Nenner $\leq |S|$, welcher die Differenz zu $\frac{2^m\frac{k}{r}}{2^m}$ minimiert, der Bruch $\frac{k}{r}$ selbst ist\footnote{Für dieses Resultat verweise ich auf \cite{QC}. Die Beschreibung befindet sich auf Seite 229 und der Beweis des dazu verwendeten Theorems im Appendix 4}. Wenn wir also mit Hilfe von Kettenbrüchen den Näherungsbruch mit dem grössten Nenner $\leq |S|$ nehmen, muss dieser Näherungsbruch $\frac{k}{r}$ sein.
\paragraph{}
Nun kann es sein, dass $\text{ggT}(k, r) = g \neq 1$ gilt, somit der Bruch mit $g$ gekürzt wird, und wir so an Stelle von $r$ die Zahl $\frac{r}{g}$ zurückbekommen. Wir können aber durch Anwendung von $f^r$ überprüfen, ob $r$ tatsächlich die Periode ist, da für $r' < r$ die Gleichung $f^{r'}(x) = x$ nicht gelten kann. Um den richtigen Wert $r$ zu erhalten, können wir die Prozedur wiederholen. Wiederholen wir sie $2\log(|S|)$ mal, erhalten wir mit hoher Wahrscheinlichkeit mindestens einmal die korrekte Periode. Den Beweis dazu kann man in \cite{QC}, Seiten 229ff., nachlesen.

\section{Die Ordnung von Zahlen bestimmen}
Der Algorithmus von Shor ist deshalb so schnell, da mit Hilfe des quantenbasierten Teils des Algorithmus die Ordnung einer Zahl effizient bestimmt werden kann. Die Ordnung einer Zahl $a$ modulo einer Zahl $n$, geschrieben als $\text{ord}_n(a)$, ist die kleinste positive Zahl $r$, sodass $a^r \equiv 1 \pmod{n}$ gilt. Dabei muss $\text{ggT}(a, n) = 1$ gelten, sonst kann dieses $r$ nicht existieren (zum Beispiel gibt es kein $r$ mit $2^r \equiv 1 \pmod{8}$). Wenn diese Bedingung jedoch erfüllt ist, besagt der Satz von Euler-Fermat, dass $a^{\phi(n)} \equiv 1 \pmod{n}$ gilt, wobei $\phi(n)$ die eulersche Phi-Funktion ist. $\phi(n)$ ist aber nicht zwingend der kleinste mögliche Exponent. 

\paragraph{}
Wenn wir $\text{ggT}(a, n) = 1$ voraussetzen, ist die Restklasse kongruent zu $a$ modulo $n$ ein Element der primen Restklassengruppe $(\mathbb {Z}/n\mathbb {Z} )^{\times}$. Insbesondere existiert ein inverses Element $a^{-1}$ modulo $n$. Wir rechnen nun in $(\mathbb {Z}/n\mathbb {Z})^{\times}$.
\paragraph{}
Sei $f$ definiert als $f(x) := ax$. Da diese Funktion invertierbar ist (die inverse Funktion ist $f^{-1}(x) = a^{-1}x$), ist sie bijektiv. Wir stellen nun fest, dass die Zahl $r = \text{ord}_n(a)$ die kleinste Zahl ist, für welche $f^r(1) = a^r\cdot 1 = 1$ gilt. Da $f$ bijektiv ist, kann man sie auf Quantencomputern implementieren. Tatsächlich haben wir diese Funktion bereits implementiert: Sie ist nichts anderes als die in 3.5 beschriebene modulare Multiplikation.

\paragraph{}
Sei $U_f$ die Quantenoperation, die $f$ auf Quantencomputern implementiert. Nun können wir der Periodenabschätzungsfunktion diesen Quantenoperator und als Startwert die Zahl $1$, bzw. den Zustand $\ket{1}$ übergeben, und sie berechnet für uns die kleinste Zahl $r$, sodass $f^r(1) = 1$ gilt, was tatsächlich die gesuchte Ordnung ist. Um klassisch effizient zu überprüfen, ob das gefundene $r$ wirklich die Ordnung von $a$ ist, können wir die Binäre Exponentiation\footnote{Siehe auch \cite{clrs}, Abschnitt 31.6, \grqq Raising to powers with repeated squaring\grqq{}} verwenden.
\paragraph{}
Es stellt sich nun die Frage, ob sich $U_f^{2^k}$ effizient implementieren lässt. Dazu stellen wir fest, dass $U_f^{2^k}$ nichts anderes als die Quantenoperation ist, welche $f^{2^k}$ implementiert. Gleichzeitig gilt $f^{2^k}(x) = a^{2^k}x$. Dies wiederum ist nichts anderes als eine Modulare Multiplikation mit der Zahl $a^{2^k}$, wobei sich die Zahl $a^{2^k} \pmod{n}$ klassisch effizient berechnen lässt (auch dazu können wir wieder die Binäre Exponentiation verwenden). Somit können wir für jedes $k$ die Quantenoperation $U_f^{2^k}$ mit nur einer modularen Multiplikation implementieren.
\paragraph{}
Fassen wir das Gesagte zusammen: Wir können die im vorangehenden Abschnitt erklärte Periodenabschätzungsfunktion verwenden, um die Ordnung einer Zahl $a$ modulo einer Zahl $n$ effizient abzuschätzen, falls $\text{ggT}(a, n) = 1$ gilt.

\section{Das Ziel - Der Shor-Algorithmus}
\subsection{Der Algorithmus}
Im Grunde genommen erlaubt uns der Algorithmus von Shor nur, einen nicht-trivialen Teiler $d$ von $n$ zu finden. Dies lässt sich jedoch leicht zu einem Algorithmus erweitern, welcher Zahlen faktorisiert: Solange $n$ keine Primpotenz ist, finde einen nicht-trivialen Teiler $d$ von $n$ und faktorisiere rekursiv die beiden Zahlen $d$ und $\frac{n}{d}$. Versuchen wir nun, einen nicht-trivialen Teiler von $n$ zu finden.

\paragraph{}
In einem ersten Schritt überprüfen wir, ob $n$ durch $2$ teilbar und / oder eine Primpotenz ist. Letzteres lässt sich überprüfen, indem man für jede natürliche Zahl $s \leq \log(n)$ überprüft, ob eine Zahl $p$ mit $p^s = n$ existiert. Diese Zahl $p$, falls sie existiert, kann man mit einer binären Suche finden (auch wenn $p$ keine Primzahl ist, haben wir trotzdem bereits einen Faktor gefunden und können terminieren).
\paragraph{}
Als nächstes wählen wir eine zufällige Zahl $a$ mit $1 < a < n - 1$. Falls $\text{ggT}(a, n) \neq 1$ gilt, hat man bereits einen nicht-trivialen Teiler gefunden, den man zurückgeben kann. Andernfalls bestimmen wir mit Hilfe der Periodenabschätzung wie in Kapitel 4.5 beschrieben die Ordnung der Zahl $a$ modulo $n$. Dies ist der einzige Schritt des Algorithmus, bei dem wir Quantencomputer benötigen. Sei diese Ordnung $r$. Falls $r$ ungerade ist oder $a^\frac{r}{2} \equiv -1 \pmod{n}$ gilt, beginnen wir wieder von vorne. Die Wahrscheinlichkeit, dass wir wieder von vorne beginnen müssen, ist dabei aber kleiner als $\frac{1}{2^m}$,  wobei $m$ die Anzahl verschiedener Primteiler von $n$ ist. Den Beweis dazu kann man in \cite{QC} Seiten 233ff. nachlesen.
\paragraph{}

Wir sind nun im Besitz einer Zahl $a$ mit Ordnung $r$, wobei $2 \mid r$ und $a^\frac{r}{2} \not\equiv -1 \pmod{n}$ gilt. Da $r$ die Ordnung ist, gilt auch $a^\frac{r}{2} \not\equiv 1 \pmod{n}$. Dies bedeutet, wir haben eine nicht-triviale Wurzel von $1$ modulo $n$ gefunden. Nun gilt $n \mid (a^\frac{r}{2} + 1)(a^\frac{r}{2} - 1)$, aber $n \not| \; a^\frac{r}{2} + 1$ und $n \not| \; a^\frac{r}{2} - 1$. Somit haben wir zwei Zahlen $b = a^\frac{r}{2} + 1$ und $c = a^\frac{r}{2} - 1$ mit:
\begin{align}
    n \mid bc \label{eq:1} \\
    n \not| \; b \label{eq:2} \\
    n \not| \; c \label{eq:3}
\end{align}    
Sei nun $p_0^{\alpha_0}p_1^{\alpha_1}... = n$ die Primfaktorzerlegung von $n$, $p_0^{\beta_0}p_1^{\beta_1}... = b$ diejenige von $b$ und $p_0^{\gamma_0}p_1^{\gamma_1}... = c$ diejenige von $c$. Aus (1) folgt, dass für jedes $k$ die Ungleichung $\beta_k + \gamma_k \geq \alpha_k$ stimmen muss. Gleichzeitig folgt aus (2) und (3), dass ein $i$ und ein $j$ existieren müssen, sodass $\beta_i < \alpha_i$ und $\gamma_j < \alpha_j$ gelten. Daraus folgt, dass $\text{ggT}(b, n)$ und $\text{ggT}(c, n)$ nicht $n$ sein können. Gleichzeitig müssen wegen $\beta_k + \gamma_k \geq \alpha_k$ auch $\gamma_i > 0$ und $\beta_j > 0$ gelten, woraus $\text{ggT}(b, n) > 1$ und $\text{ggT}(c, n) > 1$ folgt.
\paragraph{}
Daraus schliessen wir, dass beide Zahlen $b$ und $c$ Faktoren von $n$ enthalten. Diese können durch die Berechnung von $\text{ggT}(b, n)$ und $\text{ggT}(c, n)$ mit Hilfe des euklidischen Algorithmus extrahiert werden. Fassen wir all dies zusammen, sind wir im Besitz eines Algorithmus, der einen nicht-trivialen Teiler von $n$ findet:
\begin{enumerate}
    \item Falls $n$ durch zwei teilbar ist, gib 2 zurück und terminiere.
    \item Falls $n = p^a$ eine Primpotenz ist, gib die Primzahl $p$ zurück und terminiere.
    \item Bestimme eine zufällige Zahl $1 < a < n - 1$.
    \item Finde $g = \text{ggT}(a, n)$. Falls $g \neq 1$ ist, gib $g$ zurück und terminiere.
    \item Bestimme die Ordnung von $a$ modulo $n$ mit Hilfe des Quantenteils des Algorithmus:
    \begin{enumerate}
        \item Schätze die Phase des Operators $U_f$, der $f(x) = ax$ implementiert, mit einer Präzision von $m = 2\lceil \log_2(n) \rceil$ ab. Benutze dazu den in Kapitel 4.3 vorgestellten Algorithmus. Sei das Resultat $2^m\lambda$.
        \item Schätze den Quotienten $\frac{k}{r}$ von $\frac{2^m\lambda}{2^m}$ ab. Falls $r$ nicht die gesuchte Periode ist, gehe zurück zu (a), andernfalls gib die Periode $r$ zurück.
    \end{enumerate}
    \item Falls $r$ ungerade ist, gehe zurück zu 2. Andernfalls berechne $a^{\frac{r}{2}} \pmod{n}$. Falls dies kongruent zu $-1 \pmod{n}$ ist, gehe zurück zu 2.
    \item Berechne $b = (a^\frac{r}{2} + 1)$ oder $c = (a^\frac{r}{2} - 1)$. Gib $\text{ggT}(b, n)$ oder $\text{ggT}(c, n)$ zurück und terminiere.
\end{enumerate}
\subsection{Analyse des Algorithmus}
Keiner der heute bekannten klassischen Faktorisierungsalgorithmen ist polynomiell\footnote{Ein Algorithmus ist \textit{polynomiell} in $L$, wenn die Anzahl Operationen, die er benötigt, in $\mathcal O(L^c)$ für eine Konstante $c > 0$ ist.} in der Länge $L = \lceil \log_2(n) \rceil$ der Zahl $n$, wobei die Länge der Zahl die Anzahl Bits beschreibt, welche man benötigt, um die Zahl zu speichern. Die meisten\footnote{Es gibt auch einige Ausnahmen mit subexponentiellen Laufzeiten, wie z. B. der General Number Field Sieve Algorithmus, dessen Laufzeit man heuristisch auf etwa $\exp\left(c(\log \, n)^\frac{1}{3}(\log \, \log n)^\frac{2}{3}\right)$ abschätzen kann. Für eine genauere Beschreibung verweise ich auf \cite{gnfs}} bekannten klassischen Algorithmen sind polynomiell in $n \geq 2^{L-1}$ und werden somit als exponentielle Algorithmen bezeichnet. Der Algorithmus von Shor hingegen ist polynomiell in $L$ und deshalb viel effizienter als alle bekannten Algorithmen. Ich werde nun die Laufzeit\footnote{Unter der Laufzeit eines Algorithmus versteht man eine Abschätzung der Anzahl Operationen, die ein Computer durchführen muss, um den Algorithmus auszuführen.} meiner Implementierung analysieren.

\paragraph{}

Ob $n$ durch zwei teilbar ist, lässt sich ganz einfach in $\mathcal O(1)$ überprüfen. Um zu überprüfen, ob $n$ eine Potenz einer anderen Zahl ist, überprüfen wir für jedes $k \in \{1, \dots, L\}$, ob eine Zahl $s$ mit $s^k = n$ existiert. Ob diese Zahl existiert, können wir herausfinden, indem wir zuerst die grösste Zahl $s'$ mit $(s')^k \leq n$ mit Hilfe einer binären Suche finden, und dann überprüfen, ob $(s')^k = n$ gilt. $(s')^k$ lässt sich in $\mathcal{O}(\log(k))$ mit der binären Exponentiation berechnen. Somit brauchen wir $\mathcal{O}(L^2 \, \log \, L)$ Schritte, um überprüfen zu können, ob $n$ eine Potenz einer anderen Zahl ist, wobei ein Faktor $L$ für die $L$ Repetitionen und ein Faktor $L$ von der binären Suche kommt.

\paragraph{}
Wenn $n$ eine der beiden Kriterien erfüllt, terminieren wir. Deshalb können wir an dieser Stelle annehmen, dass $n$ mindestens zwei verschiedene Primteiler besitzt. Dies führt dazu, dass die Chance, bei einem zufälligen $a$ eine Periode $r$ zu erhalten, die gerade ist und die $a^\frac{r}{2} \not\equiv -1 \pmod{n}$ erfüllt, grösser oder gleich als $\frac{1}{2}$ ist. Somit wird im Erwartungswert $\mathcal O(1)$ mal eine zufällige Zahl generiert und deren Periode berechnet. Genau so oft wird auch die Periodenabschätzungsfunktion aufgerufen. 

\paragraph{}
Die Periodenabschätzungsfunktion hat ein Zählerregister der Grösse $2L$ und ruft somit die modulare Multiplikation $2L \in \mathcal O(L)$ auf. Auch wendet sie die $QFT$ einmal an, diese ist jedoch im Vergleich mit der Laufzeit der aufgerufenen Multiplikationen vernachlässigbar. Jede davon ruft $2L \in \mathcal O(L)$ mal die Addition auf, welche zwei Fouriertransformationen zu $\mathcal O(L^2)$ Gatteroperationen anwendet. Rechnen wir alle diese Aufrufe zusammen, ergibt sich eine gesamte Gatterzahl von $\mathcal O(L^4)$ für einen Aufruf der Periodenabschätzungsfunktion. Somit ist die Gesamtlaufzeit des Algorithmus in $\mathcal O(L^4)$. Würde man die Addition mit Carry-Qubits implementieren, könnte man diese Zahl auf $\mathcal O(L^3)$ reduzieren.

\paragraph{}

Ich möchte nun einen Vergleich anstellen, in welchem wir der Einfachheit halber die konstanten Faktoren weglassen und nur ungefähr rechnen: Nehmen wir eine Zahl der Länge $L = 200$. Diese Zahl hat ungefähr $60$ Stellen im Dezimalsystem. Rechnen wir nun dieses Beispiel durch, so braucht der naive klassische Algorithmus, der alle Zahlen bis zu $\sqrt{n}$ als Teiler ausprobiert, etwa $2^{100} \approx 10^{30}$ Operationen, wofür ein moderner Computer ungefähr $10^{13}$ Jahre benötigt, unter der Annahme, dass ein moderner Computer etwa $10^9$ Operationen pro Sekunde ausführen kann\footnote{Laut den Prozessorspezifikationen von AMD \cite{amdpr} haben deren Prozessoren Schaltfrequenzen von über 4 Gigahertz.}. Nimmt man auch an, dass das Universum etwa $1.4 \cdot 10^{10}$ Jahre alt ist\footnote{In \cite{ageu} wurde der Kehrwert der Hubble-Konstante, mit welchem man das Alter des Universums abschätzt, auf 13.8 Milliarden Jahre eingeschätzt.}, sieht man, dass der Computer somit ungefähr $10^3$-mal so lange benötigt, wie das Universum alt ist. Betrachten wir nun meine Implementierung des Algorithmus, benötigt sie ungefähr $10^9$ Gatteroperationen. Auch wenn ein Quantencomputer nur $10^5$ Gatteroperationen pro Sekunde\footnote{In \cite{eqasm} wird mit 20 bzw. 40 für Quantengatter auf 1 bzw. 2 Qubits und 300 bis 1000 Nanosekunden pro Messung gerechnet. Mit diesen Werten kommt man auf über $10^7$ Gatteroperationen pro Sekunde. Einige Operationen muss man aber für die Fehlerkorrektur verwenden.} ausführen könnte, was meiner Meinung nach eine sehr tiefe Schätzung ist, würde er trotzdem nur einige Stunden benötigen. Würde man die Implementatierung in $\mathcal O(L^3)$ nehmen, kommt man mit den gleichen Annahmen auf einige Minuten.