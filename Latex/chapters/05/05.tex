\chapter{Ausblick}
\section{Man kann Zahlen effizient faktorisieren - was nun?}
Wir haben nun gesehen, dass man Zahlen mit Hilfe von Quantencomputern effizient faktorisieren kann. Einerseits ist dies eine grosse Errungenschaft: Shors Algorithmus demonstriert uns, wie man mit Quantencomputern exponentielle Verschnellerungen erreichen kann. Auf der anderen Seite birgt diese Errungenschaft auch Gefahren. Zum Beispiel können Verschlüsselungsverfahren geknackt werden, wenn man Zahlen faktorisieren kann. Ich möchte hier kurz vorzeigen, wie man das RSA-Verschlüsselungsverfahren knacken kann, wenn man Zahlen effizient faktorisieren kann. Zu einer kurzen Beschreibung verweise ich auf \cite{rsaintro} oder für die originale Arbeit auf \cite{rsaorig}. 

Sei $n = pq$ der Modulus der Verschlüsselung und $e$ der öffentliche Schlüssel. Da wir nun $n = pq$ faktorisieren können, können wir $\phi(n) = \phi(p)\cdot\phi(q) = (p - 1)(q - 1)$ berechnen. Damit können wir den privaten Schlüssel $d \equiv e^{-1} \pmod{\phi(n)}$ berechnen, denn wir wissen, dass $m^{e\cdot d} \equiv m^{e\cdot e^{-1} \pmod{\phi(n)}} \equiv m^1 \equiv m \pmod{n}$ gilt, wobei die letzte Äquivalenz wegen des Satzes von Euler-Fermat gilt.

Da nun Verschlüsselungen, die auf der Schwierigkeit der Faktorisierung beruhen, geknackt werden können, steht das Forschungsgebiet der Kryptographie vor neuen Herausforderungen. Das neue Gebiet, welches sich mit der Kryptographie im Zeitalter der Quantencomputer befasst, nennt sich Post-Quantum-Kryptographie.

\section{Quantencomputer - wie bald?}
Wie bereits angesprochen, gibt es verschiedene Möglichkeiten, Quantencomputer physikalisch zu realisieren. Technik-Firmen und Forschungsinstitute versuchen schon länger, Quantencomputer zu bauen. Bereits im Jahr 1998 gelang es zwei Forschern aus Oxford, einen Quantenalgorithmus für Deutschs Problem \cite{qimpdj}, und drei amerikanischen Forschern, Grovers Algorithmus experimentell zu realisieren \cite{qimpgr}. Bereits im Jahr 2001 folgte darauf die erste experimentelle Realisation des Shor-Algorithmus \cite{ShorImp}, mit welcher die Zahl 15 faktorisiert werden konnte.

Über die Jahre entwickelten sich Quantencomputer immer weiter. Zum Zeitpunkt dieser Arbeit haben Tech-Firmen wie Google, IBM, Intel etc. Quantenprozessoren mit bis zu 72 Qubits (Googles Bristlecone \cite{gbc}). Im Jahr 2019 legte Google einen Artikel \cite{GSP} vor, laut welchem sie die Quantenüberlegenheit erreicht hätten. Dies bedeutet, dass sie auf einem Quantencomputer etwas effizient berechnet hätten, was auf einem klassischen Computer nicht effizient berechenbar wäre. Dies löste prompt einen Disput aus und die IBM zweifelte in einem Blog die Quantenüberlegenheit an \cite{IBM}. Während im Bericht von Google behauptet wurde, ein klassischer Computer würde 10000 Jahre benötigen, wurde in IBMs Blog behauptet, dass ein klassischer Computer dies in 2.5 Tage tun könne. Der Artikel von Google blieb unpubliziert. Trotzdem ist es meiner Meinung nach beeindruckend, dass Googles Quantencomputer für diese Aufgabe nur etwa 200 Sekunden benötigt, während ein Supercomputer 2.5 Tage benötigt.

Die IBM selbst jedoch hat auch grosse Pläne. Im September 2020 hat sie in ihrer Quantum Roadmap \cite{IBMrm} angekündigt, dass sie bis im Jahr 2023 einen Quantencomputer mit 1121 Qubits bauen und somit die Qubit-Zahlen sprengen möchte.

Wenn die IBM diesen Plan durchziehen kann und Google sowie die anderen Tech-Firmen auch in naher Zukunft so grosse Quantencomputer bauen, denke ich, dass die Zeit, in der quantenbasierte Supercomputer schwierige Berechnungen übernehmen werden, nicht mehr weit entfernt ist.