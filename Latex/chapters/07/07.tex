\chapter{Anhang}
\section{Mathematische Symbole}
\begin{description}
    \item[$\ket{v}$] Schreibweise des Vektors $v$ in der Bra-Ket-Notation.
    \item[$(\ket{u}, \ket{v})$ oder $\bra{u}\ket{v}$] Skalarproukt der beiden Vektoren $\ket{u}$ und $\ket{v}$.
    \item[$\left\|\, \ket{v}\, \right\|$] Norm des Vektors $\ket{v}$
    \item[$\mathbb{R}, \mathbb{C}$] Raum der reellen beziehungsweise der komplexen Zahlen.
    \item[$\mathbb{R}^n, \mathbb{C}^n$] $n$-dimensionaler Vektorraum über die reellen beziehungsweise die komplexen Zahlen
    \item[$\overline{A}, A^T, A^\dagger$] Die komplex konjugierte, die transponierte und die adjungierte Matrix zu der Matrix $A$. Siehe 2.1.4
    \item[$I$] Die Identitätsmatrix mit der Eigenschaft $I\ket{v} = \ket{v}$ für alle $\ket{v}$. 
    \item[$H, X, Y, Z$] Das Hadamard- und die drei Pauli-Gatter. Siehe 2.2.4
    \item[$QFT$] Die Quantenfouriertransformation. Siehe 3.2
    \item[$f^k$] $k$-fache Anwendung der Funktion $f$.
    \item[$U^c$] Kontrollierte Anwendung der Operation $U$. Siehe 2.2.5.
\end{description}

\section{Literatur}
\printbibliography

\section{Redlichkeitserklärung}
