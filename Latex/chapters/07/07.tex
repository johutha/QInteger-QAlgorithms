\chapter{Anhang}
\section{Mathematische Symbole}
\begin{description}
    \item[$\ket{v}$] Schreibweise des Vektors $v$ in der Bra-Ket-Notation.
    \item[$(\ket{u}, \ket{v})$ oder $\bra{u}\ket{v}$] Skalarprodukt der beiden Vektoren $\ket{u}$ und $\ket{v}$.
    \item[$\left\|\, \ket{v}\, \right\|$] Norm des Vektors $\ket{v}$
    \item[$\mathbb{R}, \mathbb{C}$] Körper der reellen beziehungsweise der komplexen Zahlen.
    \item[$\mathbb{R}^n, \mathbb{C}^n$] $n$-dimensionaler Vektorraum über den reellen beziehungsweise den komplexen Zahlen
    \item[$\overline{A}, A^T, A^\dagger$] Die komplex konjugierte, die transponierte und die adjungierte Matrix zu der Matrix $A$. Siehe 2.1.4
    \item[$I$] Die Identitätsmatrix mit der Eigenschaft $I\ket{v} = \ket{v}$ für alle $\ket{v}$. 
    \item[$H, X, Y, Z$] Das Hadamard- und die drei Pauli-Gatter. Siehe 2.2.4
    \item[$QFT$] Die Quantenfouriertransformation. Siehe 3.2
    \item[$f^k$] $k$-fache Anwendung der Funktion $f$.
    \item[$U^c$] Kontrollierte Anwendung der Operation $U$. Siehe 2.2.5.
    \item[$\mathbb{Z} / n \mathbb{Z}$] Restklassenring der Restklassen bei der Division durch $n$. 
    \item[$(\mathbb {Z}/n\mathbb {Z} )^{\times}$] Prime Restklassengruppe der Restklassen bei der Division durch $n$, welche mit $n$ teilerfremd sind.
    \item[$|S|$] Mächtigkeit einer Menge $S$.
    \item[$a \mid b$] $a$ ist ein Teiler von $b$.
    \item[$\log(n)$] Logarithmus einer Zahl $n$. Meistens wird dabei der binäre Logarithmus gemeint, Logarithmen mit verschiedenen Basen unterscheiden sich jedoch nur durch einen Konstanten Faktor.
    \item[$f(n) \in \mathcal O(g(n))$] Es existieren $c$ und $n_0$, sodass $0 \leq f(n) \leq c\cdot g(n)$ für alle $n\geq n_0$.
\end{description}
\pagebreak
\section{Literatur}
\printbibliography[heading=none]
\pagebreak
\section{Code}
Hier eine Zusammenfassung der wichtigsten Code-Ausschnitten zu den einzelnen Kapitel.
\paragraph{}

\textbf{Quantenfouriertransformation} (zu Kapitel 3.2):
\lstinputlisting{assets/code/7/7.3.1}
\paragraph{}

\textbf{Addition auf Qubits} (zu Kapitel 3.3):
\lstinputlisting{assets/code/7/7.3.2}
\paragraph{}

\textbf{Modulare Addition auf Qubits} (zu Kapitel 3.4):
\lstinputlisting{assets/code/7/7.3.3}
\paragraph{}

\textbf{Modulare Multiplikation auf Qubits} (zu Kapitel 3.5):
\lstinputlisting{assets/code/7/7.3.4}
\paragraph{}

\textbf{Phasenabschätzung} (zu Kapitel 4.3):
\lstinputlisting{assets/code/7/7.3.5}
\paragraph{}

\textbf{Periodenabschätzung} (zu Kapitel 4.4):
\lstinputlisting{assets/code/7/7.3.6}
\paragraph{}

\textbf{Ordnungsabschätzung - Quantenbasierter Teil von Shors Algorithm} (zu Kapitel 4.6):
\lstinputlisting{assets/code/7/7.3.7}
\paragraph{}
\section{Redlichkeitserklärung}
